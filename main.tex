\documentclass[b6paper, oneside]{memoir}
\usepackage[margin=0.5in]{geometry}
\usepackage{fontspec}

\setmainfont{Adobe Garamond Pro}

\begin{document}

\PoemTitle{}
\begin{verse}
    He was the plaintiff, I was the accused, \\
    Fortune the judge, who did the trial commence, \\
    His wits and mine the jury (stark unused), \\
    \textit{Cupid} the suborner, love the offence. \\
    Once the accuser had put forth his plaint \\
    (Would that he was some untapped ruth dissembling!), \\
    The justice bade me speak without constraint, \\
    But woe had tied my tongue, and left me trembling. \\
    `What's this?' quoth she, `Why do you quake to speak?' \\
    I answered not, although my heart was bawling. \\
    She grew displeased, and in a fit of pique, \\
    Condemned me to a penalty appalling: \\
    \vin `Since to refrain from speech you thus presume, \\
    \vin The silence of your love shall be your doom.'
\end{verse}

\PoemTitle{}
\begin{verse}
    The faults that mar you, and purloin my peace \\
    Are all within an ashen Thread contained, \\
    To wear which cursed cord you never cease \\
    That in a ring of vice your worth lies chained. \\
    No dreadful cankerworm could e'er surround \\
    The slender waist of any lily white \\
    Such that its beauty, being in baseness bound, \\
    Would to my eyes comprise a sadder sight. \\
    Must Prejudice, and Perjury, and Pride \\
    Be sheltered, whilst Pity's no pity shown? \\
    O let that rightful queen in you reside; \\
    Oust her usurpers, and their wiry throne. \\
    \vin By scores of tongues your praises shall be bred \\
    \vin If you release your worth and shed the Thread.
\end{verse}

\PoemTitle{}
\begin{verse}
    Wherefore do others harbour no concern \\
    For hapless virtue, which is thus immured? \\
    Must I alone that wretched Thread discern, \\
    Whilst to its grasp the rest remain inured? \\
    With what persuasive spells have you devised \\
    The hiding of your worthiness to hide? \\
    And in what garments are your flaws disguised \\
    As noble traits, which all but me misguide? \\
    I see! Your real merits are those charms, \\
    Which thus contribute to their own confinement; \\
    Your beauty is that garb, which fools disarms, \\
    And makes them misconceive your rare refinement. \\
    \vin Would that I were as ignorant as they! \\
    \vin For my perception leads to my decay.
\end{verse}

\PoemTitle{}
\begin{verse}
    No more, dear friends, relay to me no more \\
    His noxious taunts, those beasts who love to maul me; \\
    I, who have died a thousand times before, \\
    May not revive if further deaths befall me. \\ 
    If you espy them stumbling o'er his lips, \\
    Then in your sturdy minds immure them all; \\
    Maintain this shield, this nescient eclipse \\
    Till age permits me to dismiss his gall. \\
    For if I should detect their presence there, \\
    I'll make you free them with beseeching guile, \\
    And let them drive me to my dark despair; \\
    So much I crave the truth, though truth be vile. \\
    \vin I hate my peace, and yearn for ceaseless pain; \\
    \vin So much I please to die and die again.
\end{verse}

\PoemTitle{}
\begin{verse}
    As brave \textit{Prometheus}, having stol'n that fire \\
    Which by the Thunderer was from mortals hidden, \\
    Did bear the brunt of his horrendous ire: \\
    An endless retribution, torture-ridden; \\
    As, every day, the liver of the thief \\
    Was into fodder for \textit{Jove's} eagle made, \\
    And, every night, grew back (much to his grief) \\
    And blissful death was evermore delayed; \\
    So is my heart by tyrant Love ensnared: \\
    He shoots at it, heals it, then shoots again, \\
    And yet the heart of my heart's thief is spared \\
    From such recurrent, mischief-laden pain. \\
    \vin Here is a proof that \textit{Cupid} suffers blindness: \\
    \vin He dooms the robbed, and pardons thieves with kindness.
\end{verse}
    
\end{document}
