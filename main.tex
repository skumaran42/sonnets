\documentclass[a5paper, 12pt, oneside]{memoir}
\usepackage[top=0.6in, bottom=0.6in, left=0.25in, right=0.25in]{geometry}
\usepackage{fontspec, pgfornament, soul, lettrine, Zallman}

\hfuzz=999pt
\setmainfont{Adobe Garamond Pro}
\setlength{\DefaultFindent}{1pt}
\renewcommand{\LettrineFontHook}{\Zallmanfamily}

\begin{document}

\begin{titlingpage}
\null  % Empty line
\nointerlineskip  % No skip for prev line
\vfill
\let\snewpage \newpage
\let\newpage \relax

\centering
{\HUGE \so{AMORINI}} \\[5pt]
{\Large Containing fifty Sonnets} \\[10pt]
{\Large\itshape Santhosh Kumaran}

\let \newpage \snewpage
\vfill 
\break % page break
\end{titlingpage}

\PoemTitle{}
\vindent0pt
\begin{verse}
    \lettrine{F}{air} \textit{Psyche}, chancing to descry my love, \\>[-6pt]
    Went mad with lust, and longed for him amain, \\
    And left poor \textit{Cupid} in his realm above \\
    Who had implored her to remain in vain. \\
    But when the flighty goddess `gan to woo him, \\
    He spurned her praise, and laughed her love to scorn. \\
    She begged and sobbed, unable to eschew him, \\
    But left at length, dejected and forlorn. \\
    And \textit{Cupid}, with a vengeful envy burning, \\
    Let fly his darts (with lethal love enfierced) \\
    Against the object of her newfound yearning \\
    But ah, his flinty heart could not be pierced. \\
    \vin Nay, it deflected each and every dart \\
    \vin Into a milder prey: my heedless heart.
\end{verse}

\PoemTitle{}
\vindent0pt
\begin{verse}
    \lettrine{W}{hat} holy rapture set my heart aflame \\>[-6pt]
    When first it felt that wanton archer's darts! \\
    Just as the tendrils on the sundew's frame, \\
    Whose saccharine glue such hope to bugs imparts. \\
    How fervently I begged their roguish lord \\
    To send forth more and master me with power! \\
    So much I loved the love he did afford \\
    That I myself built in my heart his bower. \\
    But soon the leaf which lavished me with nectar \\
    Enclosed me whole, and changed its sweets to gall, \\
    Reducing me unto a lifeless spectre \\
    And feeding on my smarting tears withal. \\
    \vin Alas, it is too late to fly away; \\
    \vin Nought can I do but watch myself decay.
\end{verse}

\PoemTitle{}
\vindent0pt
\begin{verse}
    \lettrine{H}{e} was the plaintiff, I was the accused, \\>[-6pt]
    Fortune the judge, who did the trial commence, \\
    His wits and mine the jury (stark unused), \\
    \textit{Cupid} the suborner, love the offence. \\
    Once the accuser had put forth his plaint \\
    (O that he was some untapped ruth dissembling!), \\
    The justice bade me speak without constraint, \\
    But woe had tied my tongue, and left me trembling. \\
    `What's this?' quoth she, `Why do you quake to speak?' \\
    I answered not, although my heart was bawling. \\
    She grew displeased, and in a fit of pique, \\
    Condemned me to a penalty appalling: \\
    \vin `Since to refrain from speech you thus presume, \\
    \vin The silence of your love shall be your doom.'
\end{verse}

\PoemTitle{}
\vindent0pt
\begin{verse}
    \lettrine{A}{s} brave \textit{Prometheus}, having stol'n that fire \\>[-6pt]
    Which by the Thunderer was from mortals hidden, \\
    Did bear the brunt of his horrendous ire: \\
    An endless retribution, torture-ridden; \\
    As, every day, the liver of the thief \\
    Was into fodder for \textit{Jove's} eagle made, \\
    And, every night, grew back (much to his grief) \\
    And blissful death was evermore delayed; \\
    So is my heart by \textit{Cupid's} darts ensnared, \\
    Then healed by treacherous hope, then shot again. \\
    And yet the heart of my heart's thief is spared \\
    From such recurrent, mischief-laden pain. \\
    \vin Such is love's law: whereas the thief goes free, \\
    \vin His victim must endure the penalty. 
\end{verse}

\PoemTitle{}
\vindent0pt
\begin{verse}
    \lettrine{T}{he} faults that mar you, and purloin my peace \\>[-6pt]
    Are all within an ashen Thread contained, \\
    To wear which cursed cord you never cease \\
    That in a ring of vice your worth lies chained. \\
    No dreadful cankerworm could e'er surround \\
    The slender waist of any lily white \\
    Such that its beauty, being in baseness bound, \\
    Would to my eyes comprise a sadder sight. \\
    Must Prejudice, and Perjury, and Pride \\
    Be sheltered, whilst Pity's no pity shown? \\
    Fair! Let that rightful queen in you reside; \\
    Oust her usurpers, and their wiry throne. \\
    \vin By scores of tongues your praises shall be bred \\
    \vin If you unbind your worth and shed the Thread.
\end{verse}

\PoemTitle{}
\vindent0pt
\begin{verse}
    \lettrine{W}{herefore} do others harbour no concern \\>[-6pt]
    For hapless virtue, which is thus immured? \\
    Must I alone that wretched Thread discern, \\
    Whilst to its grasp the rest remain inured? \\
    With what persuasive spells have you devised \\
    The hiding of your worthiness to hide? \\
    And in what garments are your flaws disguised \\
    As noble traits, which all but me misguide? \\
    I see: Your real merits are those charms, \\
    Which thus contribute to their own confinement; \\
    Your beauty is that garb, which fools disarms, \\
    And makes them misconceive your rare refinement. \\
    \vin O that I were as ignorant as they! \\
    \vin For my perception leads to my decay.
\end{verse}

\PoemTitle{}
\vindent0pt
\begin{verse}
    \lettrine{N}{o} more, dear friends, relay to me no more \\>[-6pt]
    His noxious jeers, those beasts who love to maul me; \\
    I, who have died a thousand times before, \\
    May not revive if further deaths befall me. \\ 
    If you espy them stumbling o'er his lips, \\
    Then in your sturdy minds confine them all; \\
    But make no mention of this blest eclipse \\
    Till age permits me to dismiss his gall. \\
    For if I should detect their presence there \\
    I'll make you free them with beseeching guile, \\
    And drown once more in deathly, dark despair \\
    To see them maim my feeble peace the while. \\
    \vin So much I crave to know the direful truth \\
    \vin That I would fain with deaths befoul my youth.
\end{verse}

\PoemTitle{}
\vindent0pt
\begin{verse}
    \lettrine{L}{o}, in a hall illumed with honour's light, \\>[-6pt]
    I stand content before a friendly horde \\
    Which cheers amain, applauding in delight \\
    As I receive some coveted award. \\
    But my eyes roam the throng in search of him, \\
    For whose ambrosian praise I languish most, \\
    And find him full of scorn, unmoved and grim; \\
    Not clapping once, rebelling 'gainst the host. \\
    At once the plaudits cease, the praise subsides, \\
    And doleful darkness fills the silent room; \\
    My bold resolve is wholly choked besides, \\
    And lucent rapture finds its inky tomb. \\
    \vin All honours fade to dust if he disdains them, \\
    \vin And yet my eyes leave not the sight that pains them.
\end{verse}

\PoemTitle{}
\vindent0pt
\begin{verse}
    \lettrine{M}{irth} seems unto me like some foreign tongue, \\>[-6pt]
    And yet I know it is the common speech, \\
    Whose wanton words console both old and young \\
    But seem to sneer at my oblivious breach. \\
    How can I learn it? Why do I not know it? \\
    Ah, e'en the worst of halfwits prove so fluent! \\
    Such umbrage plagues me (though I cannot show it) \\
    To see myself astray, and them pursuant! \\
    But I forget! I spoke it once with skill, \\
    'Twas he, that scholar, who had taught me how. \\
    But since he ceased his posting to fulfil, \\
    I ceased to speak it, and it shuns me now. \\
    \vin Why must you, sir, so gladly teach the rest \\
    \vin But oust the pupil who so well progressed?
\end{verse}

\end{document}
